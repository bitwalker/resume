%
% LaTeX source of my resume
% =========================
%
% Heavily commented to to fit even LaTeX beginners (hopefully).
%
% See the `README.md` file for more info.
%
% This file is licensed under the CC-NC-ND Creative Commons license.
%


% Start a document with the here given default font size and paper size.
\documentclass[10pt,a4paper]{article}

% Set the page margins.
\usepackage[a4paper,margin=0.75in]{geometry}

% Setup the language.
\usepackage[english]{babel}
\hyphenation{Some-long-word}

% Makes resume-specific commands available.
\usepackage{resume}




\begin{document}  % begin the content of the document
\sloppy  % this to relax whitespacing in favour of straight margins


% title on top of the document
\maintitle{Paul Schoenfelder}{April 12, 1987}{Last update on \today}

\nobreakvspace{0.3em}  % add some page break averse vertical spacing

% \noindent prevents paragraph's first lines from indenting
% \mbox is used to obfuscate the email address
% \sbull is a spaced bullet
% \href well..
% \\ breaks the line into a new paragraph
\noindent\href{mailto:paulschoenfelder@fastmail.com.com}{paulschoenfelder\mbox{}@\mbox{}fastmail.com}\sbull
952-905-4095\sbull
\href{http://bitwalker.org}{bitwalker.org}
\\
Saratoga Springs\sbull
New York\sbull
12866

\spacedhrule{0.9em}{-0.4em}  % a horizontal line with some vertical spacing before and after

\roottitle{Summary}  % a root section title

\vspace{-1.3em}  % some vertical spacing
\begin{multicols}{2}  % open a multicolumn environment
\noindent \emph{I'm an experienced software engineer with a love for open source, languages and distributed computing, and a passion for developing high-quality software.}
\\
\\
I started out in IT as a network engineer, working with Cisco equipment, but it wasn't long before I had delved in to writing scripts for automation. Once I had a taste for code, I knew it was where I belonged, and I was relentless in pursuing an open developer position at the same company. I was given the opportunity, and I've been a software engineer ever since.

I was thrown in to the world of software engineering, and had to learn fast. I've soaked up books, blogs, talks, whatever I could. I have become proficient in numerous languages, technologies, and practices along the way, and achieved mastery in many of them.

I'm active in open source. I experiment with embedded development. I love functional programming and the challenges of concurrency and distribution. But most of all, I love building software.
\end{multicols}


\spacedhrule{0em}{-0.4em}

\roottitle{Professional Experience}

\headedsection  % sets the header for the section and includes any subsections
  {\href{http://www.ockam.io}{Ockam}}
  {\textsc{Remote}} {%
  \headedsubsection
    {Architectural Engineer}
    {Jan \apo20 -- present}
    {\bodytext{My role at Ockam is to build out the cloud services platform, in particular, the messaging and routing layer.}}
}

\headedsection  % sets the header for the section and includes any subsections
  {\href{http://www.dockyard.com}{DockYard}}
  {\textsc{Remote}} {%
  \headedsubsection
    {Architectural Engineer}
    {Oct \apo17 -- Jan \apo20}
    {\bodytext{My role at DockYard took two shapes: first as software architect on client projects, primarily focused on deployment and infrastructure; second as technical lead on various R\&D projects. I was originally hired to work on my open source projects, namely Distillery, and get releases integrated into Elixir Core, which happened in Elixir 1.9. The last year of my time at DockYard was spent working on Lumen, an ahead-of-time compiler for Erlang and Elixir that targets WebAssembly.}}
}

\headedsection  % sets the header for the section and includes any subsections
  {\href{http://www.fireeye.com}{FireEye}}
  {\textsc{Remote}} {%
  \headedsubsection
    {Senior Staff Software Engineer}
    {Jan \apo17 -- Oct \apo17}
    {\bodytext{My role at FireEye was focused on the implementation of a concurrent, distributed, throughput-heavy workflow and data processing engine for security orchestration, built in Elixir.}}
}

\headedsection  % sets the header for the section and includes any subsections
  {\href{http://www.exosite.com}{Exosite}}
  {\textsc{Minneapolis, Minnesota}} {%
  \headedsubsection
    {Senior Web and Infrastructure Engineer}
    {Oct \apo15 -- Jan \apo17}
    {\bodytext{Lead developer on some of Exosite's most critical core services, built on Erlang and Elixir, with a few in Go (with parts written in C). Became very proficient with an array of AWS services, as well as  Kubernetes in order to implement an OpenShift-based hosting platform for customer and internal applications. The applications I built were distributed, highly concurrent infrastructure services, with all the resultant challenges. I am heavily involved in the Erlang/Elixir community, and gave two talks, one at ElixirConf 2015, and one at ElixirConf 2016.}}
}

\headedsection  % sets the header for the section and includes any subsections
  {\href{http://www.nerdery.com}{The Nerdery}}
  {\textsc{Bloomington, Minnesota}} {%
  \headedsubsection
    {Senior Software Engineer}
    {Mar \apo13 -- Oct 15}
    {\bodytext{Lead developer for a number of projects both large and small, managing teams of 2 to 10 devs. C\# and Scala work primarily, but some F\# as well. Built a number of open source libraries for the Elixir community, including contributions to the language itself. Focused on growing my functional programming, cryptography, and concurrent/distributed systems skills in my spare time.}}
}

\headedsection
  {United States Air Force}
  {\textsc{Madison, Wisconsin}} {%
  \headedsubsection
    {Avionic Systems Journeyman}
    {Apr \apo09 -- Apr \apo15}
    {\bodytext{While in the military, I achieved the rank of Senior Airman while working on Lockheed Martin F-16C/D fighter aircraft. I was responsible for maintaining the avionics systems, which involved understanding software, electronics theory, and numerous complicated subsystems of the aircraft.}}
}


\vspace{-0.2em}
\begin{center}
  \emph{\small You may refer to my \href{http://www.linkedin.com/in/gotbones}{LinkedIn profile} for more a more complete list of my work experiences along with recommendations.}
\end{center}

\spacedhrule{-0.2em}{-0.4em}

\roottitle{Notable Open Source Projects}

\headedsection
  {\href{https://github.com/lumen/lumen}{Lumen}}{%
  \headedsubsection
    {Technical Lead/Maintainer}
    {Jan \apo19 -- present}
    {\bodytext{Lumen is an ahead-of-time compiler for Erlang and Elixir that can target a variety of architectures, of which WebAssembly is of primary interest. It can be thought of as an alternative to the BEAM virtual machine, except rather than a virtual machine, Lumen compiles to native code. In addition to the compiler, Lumen also includes a runtime that allows it to integrate with browser APIs when targeting WebAssembly. It is built in Rust and C++, and builds on top of LLVM.}}}

\headedsection
  {\href{https://github.com/bitwalker/distillery}{Distillery}}{%
  \headedsubsection
    {Author/Maintainer}
    {May \apo16 -- present}
    {\bodytext{Distillery is a release-management tool for Elixir applications, effectively a redesign/rewrite of ExRM, the original tool I wrote for the same purpose. It is the primary and officially recommended tool for deploying Elixir-based applications. It is a very active OSS project, with many contributors.}}}
    
\headedsection
  {\href{https://github.com/bitwalker/timex}{Timex}}{%
  \headedsubsection
    {Author/Maintainer}
    {Nov \apo13 -- present}
    {\bodytext{Timex is the premier date/time library for Elixir projects. It provides functionality that even most standard library packages in other languages fail to offer. It provides rich parsing/formatting facilities (including locale-awareness), timezone-aware date/time arithmetic, a modular architecture, operations on intervals, and more.}}}
    
\vspace{-0.2em}
\begin{center}
  \emph{\small My GitHub profile contains a list of the many other projects I'm involved in, feel free to take a look!}
\end{center}

\spacedhrule{-0.2em}{-0.4em}

\roottitle{Conference Talks}

\inlineheadsection
  {2019:}
  {CodeMesh LDN, ElixirConf, LoneStar Elixir}
  
\inlineheadsection
  {2018:}
  {The Big Elixir, LoneStar Elixir, CodeBEAM SF}
  
\inlineheadsection
  {Older:}
  {ElixirConf x2, LoneStar Elixir}
  
  
\roottitle{Podcasts}

\inlineheadsection
  {2020:}
  {Elixir Wizards}
  
  
\inlineheadsection
  {2019:}
  {Smart Software with SmartLogic, Elixir Talk, ElixirMix}



\roottitle{Skills}

\inlineheadsection  % special section that has an inline header with a 'hanging' paragraph
  {Languages:}
  {C, C\#, Clojure, Elixir, Erlang, F\#, Go, Javascript, Lua, OCaml, Python, Ruby, Rust, Scala, Shell}
\inlineheadsection
  {Theory:}
  {Type theory, compilers, interpreters, data structures and algorithms, distributed consensus, messaging and protocol design}
\inlineheadsection
  {Dev/Ops:}
  {Networking, administration of Linux and Windows systems, deployment automation, deployment orchestration, continuous integration with internal and external build systems, AWS services, Azure services, Kubernetes, OpenShift}
\inlineheadsection
  {Development:}
  {I've built applications ranging from simple scripts to desktop applications with graphical interfaces. Single-threaded console utilities, to distributed, heavily-concurrent backend services where performance is critical. I've worked on scripting systems, web applications, some embedded software, and designed and developed remotely upgradeable, hot-pluggable services for gateway devices. If it's something new, I make a point to learn everything I can about the domain, and then dive in.}
\inlineheadsection
  {Non-technical skills:}
  {Can speak beginner-level Russian, and trying to improve every day}

\roottitle{Interests}

\inlineheadsection
  {Non-exhaustive and in no particular order:}
  {programming language design and theory, cryptography, artificial intelligence, game development, music, snowboarding, homebrewing, motorcycles, the outdoors, and more! I love to learn, so there is almost always something new I'm picking up.}


\end{document}
